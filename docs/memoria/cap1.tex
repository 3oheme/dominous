% -*-cap1.tex-*-
% Este fichero es parte de la plantilla LaTeX para
% la realización de Proyectos Final de Carrera, protejido
% bajo los términos de la licencia GFDL.
% Para más información, la licencia completa viene incluida en el
% fichero fdl-1.3.tex

% Copyright (C) 2009 Pablo Recio Quijano 

\section{¿Por qué un simulador de dominó}

A la hora de embarcarse en el desarrollo de un Proyecto Fin de Carrera, la primera duda es obvia: ¿Sobre qué va a versar mi proyecto?.

El Proyecto Fin de Carrera es el culmen a un largo período de aprendizaje, exámenes y vivencias y experiencias, y por estas razones la elección de una temática para el proyecto es compleja, ya que tenemos diferentes necesidades, limitaciones e impulsos:
\begin{enumerate}
    \item Por una parte el proyecto es una parte más de nuestros estudios universitarios, que debemos solventar con éxito, y esta característica nos puede llevar a buscar un proyecto más recortado o limitado en cuanto a requerimientos de tiempo y conocimiento
    \item Pero por otra parte nuestra faceta de ingenieros nos impulsa a aprender, a enfrentarnos con nuevos problemas y dificultades, a 
\end{enumerate}
