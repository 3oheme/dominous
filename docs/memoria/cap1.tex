% -*-cap1.tex-*-
% Este fichero es parte de la plantilla LaTeX para
% la realización de Proyectos Final de Carrera, protegido
% bajo los términos de la licencia GFDL.
% Para más información, la licencia completa viene incluida en el
% fichero fdl-1.3.tex

% Copyright (C) 2009 Pablo Recio Quijano 

%\section{¿Por qué un simulador de dominó?}


%A la hora de embarcarse en el desarrollo de un Proyecto Fin de Carrera, la primera duda es obvia: ¿Sobre qué va a versar mi proyecto?.\\

El Proyecto Fin de Carrera es el culmen a un largo período de aprendizaje, exámenes, vivencias y
experiencias, y por estas razones la elección de una temática para el proyecto es compleja, ya que
tenemos diferentes necesidades, limitaciones y deseos:
\begin{enumerate}
    \item Por una parte el proyecto es una facción más de nuestros estudios universitarios, que
            debemos solventar con éxito, y esta circunstancia nos puede llevar a buscar un proyecto
            más recortado o limitado en cuanto a requerimientos de tiempo y conocimiento.
    \item Pero por otra parte nuestra faceta de ingenieros nos impulsa a aprender, a enfrentarnos
            con nuevos problemas y dificultades, a atacar ejercicios mentales duros e interesantes
            para hacer sudar nuestra mente.
\end{enumerate}

\section{Objetivo}

Después de tantear varios proyectos que tenía en mente, mis tutores me presentaron la posibilidad
de embarcarme en el desarrollo de un simulador de dominó. Al principio tomé la idea un poco en broma,
ya que la temática en principio puede parecer poco tecnológica, demasiado localizada o con escaso
atractivo, pero una vez analizado, el proyecto tenía todo lo que le podía pedir:
\begin{enumerate}
    \item El apartado de Inteligencia Artificial es muy complejo, con lo cual se puede abordar de
            diferentes maneras. Es un
            problema de elevada complejidad computacional si intentamos resolverlo mediante simples
            árboles de decisión: como el juego se desarrolla dentro de un marco de conocimiento
            limitado (no conocemos las fichas de los demás jugadores) se produce una explosión
            combinatoria que nos obliga a buscar otros métodos y herramientas, como técnicas de sistemas expertos.\\
            Esta búsqueda de nuevas técnicas para la resolución de un problema concreto es la base
            misma de la Ingeniería Informática, y es un clarísimo ejemplo de Proyecto Fin de Carrera.
    \item Por otro lado el desarrollo de videojuegos relaciona multitud de aspectos que resultan
            interesantes a la hora de ser abordados, como pueden ser:
            \begin{itemize}
                \item Programación gráfica, un asunto complejo que no se suele abordar dentro del
                        plan de estudios de una Ingeniería Técnica en Informática de Gestión, y que
                        presenta un gran número de dificultades, como resoluciones de pantalla, velocidad
                        y optimización del sistema, diferencias sustanciales entre sistemas operativos,
                        tratamiento de excepciones, entre otros.
                \item Diseño de interfaces, haciendo que la aplicación sea fácil de usar, divertida,
                        sencilla y atractiva para el usuario final, y controlando diferentes opciones
                        y dispositivos de entrada.
                \item Sistema de audio, ya que nuestra aplicación debe sincronizar actividad gráfica y
                        ejecución de música y efectos de sonido.
                \item Aspecto visual, manteniendo una coherencia en cuanto a diseño gráfico de todas
                        y cada una de las páginas, secciones y menús de toda la aplicación, guardando
                        una uniformidad y buscando que la belleza de la aplicación se apoye en
                        requerimientos orientados al usuario.
            \end{itemize}
    \item Por último, tras una concienzuda búsqueda en la red, no se
      ha encontrado ningún simulador de dominó de código abierto, lo
      que creemos que es esencial para difundir el conocimiento sobre
      este juego~\cite{stallman2004software}.
\end{enumerate}

\section{Estructura de la memoria}

Esta memoria se estructurará de la siguiente forma:

\begin{enumerate}
    \item Introducción
    \item Conceptos básicos
    \item Planificación
    \item Análisis
    \item Diseño
    \item Implementación
    \item Pruebas
    \item Conclusiones
    \item Manual de instalación
    \item Manual de usuario
    \item Difusión
    \item Bibliografía y referencias
\end{enumerate}
