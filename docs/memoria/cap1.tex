% -*-cap1.tex-*-
% Este fichero es parte de la plantilla LaTeX para
% la realización de Proyectos Final de Carrera, protejido
% bajo los términos de la licencia GFDL.
% Para más información, la licencia completa viene incluida en el
% fichero fdl-1.3.tex

% Copyright (C) 2009 Pablo Recio Quijano 

\section{¿Por qué un simulador de dominó?}

A la hora de embarcarse en el desarrollo de un Proyecto Fin de Carrera, la primera duda es obvia: ¿Sobre qué va a versar mi proyecto?.

El Proyecto Fin de Carrera es el culmen a un largo período de aprendizaje, exámenes y vivencias y experiencias, y por estas razones la elección de una temática para el proyecto es compleja, ya que tenemos diferentes necesidades, limitaciones e impulsos:
\begin{enumerate}
    \item Por una parte el proyecto es una facción más de nuestros estudios universitarios, que debemos solventar con éxito, y esta circunstancia nos puede llevar a buscar un proyecto más recortado o limitado en cuanto a requerimientos de tiempo y conocimiento.
    \item Pero por otra parte nuestra faceta de ingenieros nos impulsa a aprender, a enfrentarnos con nuevos problemas y dificultades, a atacar ejercicios mentales duros e interesantes para hacer sudar nuestra mente.
\end{enumerate}

Y después de tantear varios proyectos que tenía en mente, mis tutores me presenton la posibilidad de embarcarme en el desarrollo de un simulador de dominó. Al principio tomé la idea un poco en broma, ya que la temática en principio puede parecer poco tecnológica, demasiado localizada o con escaso atractivo, pero una vez analizado, el proyecto tenía todo lo que le podía pedir:
\begin{enumerate}
    \item El apartado de Inteligencia Artificial es muy complejo, con lo cual se puede abordar de diferentes maneras, aplicando diferentes técnicas de sistemas expertos. Además es un problema de elevada complejidad computacional si intentamos resolverlo mediante simples árboles de decisión: como el juego se desarrolla dentro de un marco de conocimiento limitado (no conocemos las fichas de los demás jugadores) se produce una explosión combinatoria que nos obliga a buscar otros métodos y herramientas.
    \item Por otro lado FIXME
\end{enumerate}

\section{Estructura de la memoria}
