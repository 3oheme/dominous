% -*-memoria.tex-*-
% Este fichero es parte de la plantilla LaTeX para
% la realización de Proyectos Final de Carrera, protegido
% bajo los términos de la licencia GFDL.
% Para más información, la licencia completa viene incluida en el
% fichero fdl-1.3.tex

% Copyright (C) 2009 Pablo Recio Quijano 

%-------------------------------------------------------
% ---- Plantilla para libros / memorias PFC -----

% Realizada por Pablo Recio Quijano y Noelia Sales Montes 
% Formato de portada y primera página tomado del PFC de
% Francisco Javier Vázquez Púa, en su proyecto 'libgann'
% -------------------------------------------------------

\documentclass[a4paper,11pt,spanish]{book}

\usepackage{./estilos/estiloBase} % Basicamente son todas las
                                  % librerias usadas. En caso de que
                                  % falten librerias se van añadiendo
                                  % al fichero.
\usepackage{./estilos/colores}  % Algunos colores ya generados, para
                                % los algunos estilos más avanzados.
\usepackage{./estilos/comandos} % Algunos comandos personalizados

\graphicspath{{./imagenes/}} % Indicamos la ruta donde se encuentran
                             % las imagenes, para ahorrarnos la ruta
                             % completa, y solo modificar aquí si en
                             % un momento dado lo movemos
\usepackage{varioref} % Permite usar vref

\begin{document}

% Renombramos las figuras y las tablas
\renewcommand{\figurename}{Figura}
\renewcommand{\listfigurename}{Índice de figuras}
\renewcommand{\tablename}{Tabla}
\renewcommand{\listtablename}{Índice de tablas}
\renewcommand{\lstlistingname}{Listado}
\renewcommand*\lstlistlistingname{Índice de listados}

\pagestyle{empty}
\input{portada.tex}
\cleardoublepage

% -*-primerahoja.tex-*-
% Este fichero es parte de la plantilla LaTeX para
% la realización de Proyectos Final de Carrera, protejido
% bajo los términos de la licencia GFDL.
% Para más información, la licencia completa viene incluida en el
% fichero fdl-1.3.tex

% Fuente tomada del PFC 'libgann' de Javier Vázquez Púa

\begin{center}

  \includegraphics[scale=0.2]{logo_uca.png} \\

  \vspace{2.0cm}

  \Large{ESCUELA SUPERIOR DE INGENIERÍA} \\

  \vspace{1.0cm}

  \large{INGENIERO TÉCNICO EN INFORMÁTICA DE GESTIÓN} \\

  \vspace{2.0cm}

  \large{DOMINOUS: SIMULADOR LIBRE DE DOMINÓ} \\

  \vspace{1.0cm}

\end{center}

\begin{itemize}
\item \large{Departamento: Lenguajes y sistemas informáticos}
\item \large{Directores del proyecto: Manuel Palomo Duarte}
\item \large{Autor del proyecto: Ignacio Palomo Duarte}
\end{itemize}

\vspace{1.0cm}

\begin{flushright}
  \large{Cádiz, \today} \\

  \vspace{2.5cm}

  \large{Fdo: Ignacio Palomo Duarte}
\end{flushright}

\cleardoublepage
\pagestyle{plain}

\frontmatter % Introducción, índices ...

% -*-previo.tex-*-
% Este fichero es parte de la plantilla LaTeX para
% la realización de Proyectos Final de Carrera, protegido
% bajo los términos de la licencia GFDL.
% Para más información, la licencia completa viene incluida en el
% fichero fdl-1.3.tex

% Copyright (C) 2009 Pablo Recio Quijano 

\section*{Agradecimientos}

Gracias a todos los que han estado conmigo desde el principio del camino, gracias a los que han ajustado su paso con el mío,
y sobre todo, gracias a mi padre.

\cleardoublepage

\section*{Licencia} % Por ejemplo GFDL, aunque puede ser cualquiera

Este documento ha sido liberado bajo Licencia GFDL 1.3 (GNU Free
Documentation License). Se incluyen los términos de la licencia en
inglés al final del mismo.\\

Copyright (c) 2011 Ignacio Palomo Duarte.\\

Permission is granted to copy, distribute and/or modify this document under the
terms of the GNU Free Documentation License, Version 1.3 or any later version
published by the Free Software Foundation; with no Invariant Sections, no
Front-Cover Texts, and no Back-Cover Texts. A copy of the license is included in
the section entitled "GNU Free Documentation License".\\

\cleardoublepage

\section*{Notación y formato}

Cuando nos refiramos a un programa en concreto, utilizaremos la
notación: \programa{emacs}.\\

Cuando nos refiramos a un comando, o función de un lenguaje, usaremos
la notación: \comando{quicksort}.\\

Extractos de ficheros con texto plano o código aparecerán

\begin{lstlisting} [language=Python, numbers=left]
class nombre_de_la_regla:
    def __init__(self):
        # inicializamos los atributos que sean necesarios
        pass
    def go(self, left_tile, right_tile, board, tiles, log):
        return ficha, lado, tiempo_pensando
\end{lstlisting}

\cleardoublepage

\tableofcontents
\listoffigures
\listoftables
\lstlistoflistings


\mainmatter % Contenido en si ...

\chapter{Introducción}
% -*-cap1.tex-*-
% Este fichero es parte de la plantilla LaTeX para
% la realización de Proyectos Final de Carrera, protejido
% bajo los términos de la licencia GFDL.
% Para más información, la licencia completa viene incluida en el
% fichero fdl-1.3.tex

% Copyright (C) 2009 Pablo Recio Quijano 

\section{¿Por qué un simulador de dominó}

A la hora de embarcarse en el desarrollo de un Proyecto Fin de Carrera, la primera duda es obvia: ¿Sobre qué va a versar mi proyecto?.

El Proyecto Fin de Carrera es el culmen a un largo período de aprendizaje, exámenes y vivencias y experiencias, y por estas razones la elección de una temática para el proyecto es compleja, ya que tenemos diferentes necesidades, limitaciones e impulsos:
\begin{enumerate}
    \item Por una parte el proyecto es una parte más de nuestros estudios universitarios, que debemos solventar con éxito, y esta característica nos puede llevar a buscar un proyecto más recortado o limitado en cuanto a requerimientos de tiempo y conocimiento
    \item Pero por otra parte nuestra faceta de ingenieros nos impulsa a aprender, a enfrentarnos con nuevos problemas y dificultades, a 
\end{enumerate}


\chapter{Conceptos básicos}
% -*-cap2.tex-*-
% Este fichero es parte de la plantilla LaTeX para
% la realización de Proyectos Final de Carrera, protejido
% bajo los términos de la licencia GFDL.
% Para más información, la licencia completa viene incluida en el
% fichero fdl-1.3.tex

% Copyright (C) 2009 Pablo Recio Quijano 

\section{Conceptos básicos}

\subsection{El dominó}

\subsubsection{Historia del dominó}

El dominó, deporte y pasatiempo a un tiempo, es un juego que ha ganado adeptos a lo
largo de la historia y del tiempo. Según explica Miguel Lugo en su libro Dominó Competitivo
éste “es entretenido y fácil de aprender. Ya desde pequeños comenzamos a jugar al dominó con
frutas o con animales en lugar de hacerlo con puntos”. Además, éste ha ganado popularidad
durante los últimos años hasta el punto de televisar torneos en Europa, Norteamérica y Latinoamérica.
“En 2001 la Federación Internacional de Dominó (con sede en Barcelona, España) celebró el primer Congreso Internacional. El año siguiente se llevó a cabo el primer Campeonato del Mundo de Dominó en La Habana (Cuba). El ‘Mundial’ continúa celebrándose anualmente a partir de este punto, en España, México, Venezuela y EEUU” (Lugo, 2008: 1).

Así pues Lugo señala que el dominó nunca antes ha estado tan reconocido en todo el mundo como ahora e incluso afirma que se puede jugar por internet.

La página web http://www.domino-en-linea.com recoge que en la actualidad el dominó se juegan en todo el mundo aunque resalta que es “especialmente popular en América Latina, donde los dominós se consideran como el juego nacional de numerosos países del Caribe”. Así, también menciona los torneos anuales y los clubes locales de dominó.

“El origen del dominó parece ser muy antiguo, al menos en lo que se refiere a juegos similares y quizá pretéritos al actual” (González Sanz, 2010:22)

Cuenta Gonzalez Sanz en su libro El arte del dominó: teoría y práctica que algunos historiadores creen que puede tener origen chino, ya que éstos jugaban a un juego parecido con impresiones en piedra, y que pudo llegar a Europa a través de mercaderes y viajeros, entre los que se cita al célebre Marco Polo, los cuales y fruto de los intercambios culturales de la época, trajeron el dominó a este lado del mundo, más en concreto a la Península Itálica, primer lugar de Europa donde se ha datado la práctica de este juego.

Respecto al juego chino, Martin Gardner experto en juegos que colaboró más de 20 años en la sección “Juegos Matemáticos” de la revista Scientific American, comenta en su libro Circo Matemático que en los dominós chinos, llamado kwat p’ai, no existen piezas con caras en blanco. Éstos contienen todas las combinaciones por pares desde el (1-1)  hasta el (6-6), donde tres de los seis puntos de cara  son también rojos. Los dominós coreanos son idénticos, con la única particularidad de que en el as, el punto es mayor que en las demás piezas. En los dominós chinos, cada pieza tiene un nombre pintoresco: el (6-6) es el “cielo”; el (1-1) es “la tierra”, el (5-5) es la “flor del ciruelo”, el (6-5), “la cabeza de tigre”, etc. Los nombres de las piezas son iguales a los que reciben los 21 resultados posibles del lanzamiento de un par de dados.

Precisamente este autor Gardner explica que en la literatura occidental no hay referencias a este juego hasta mediados del siglo XVIII, en que empezaron a jugarse en Italia y Francia las primeras partidas. Desde ahí, el juego se extendió al resto del continente, y más tarde, a Inglaterra y América. En Occidente, la colección normal de piezas de dominó ha consistido siempre en 28 teselas o losetas formadas por dos cuadrados adyacentes, que contienen todos los posibles pares de dígitos, de 0 hasta 6. 

González Sanz (2010:22) señala que los dominós chinos “suelen ser en la actualidad de cartón, en vez de la madera, marfil, pasta, o ébano que es lo habitual en los occidentales, y se manejan como naipes”. Y al igual que ocurre en Europa y América, con estas piezas se realizan numerosos juegos. Respecto a los distintos juegos de dominó chinos y coreanos, este autor nos remite a Games of the Orient de Stewart Culin, obra de 1895 reimpresa en 1958 por Charles Tuttle como la mejor referencia. Asimismo apunta que no existe un dominó propio en Japón –frente al resto de países asiáticos- y dice que en este país se juega con el sistema occidental.

Sin embargo, y aunque por lo que señalan algunos autores el origen asiático del dominó es el más extendido, también existen otras versiones que atribuyen el invento del juego a árabes o egipcios, “sosteniendo que no hay pruebas para relacionar claramente el dominó europeo con el chino, pudiendo ser dos invenciones independientes y separadas en el tiempo” (2010:23). González también cuenta que se conocen otras versiones del juego como la esquimal o la coreana, con distinto número de fichas y palos al clásico.
 
TIPOLOGÍA DE DOMINÓS

Como hemos visto, a lo largo y ancho del mundo existen diferentes tipos de dominó de los que no siempre los autores coinciden con un solo origen. Aunque para catalogar tales juegos como dominó sí deben tener una serie de características en común.
Generalizando el concepto de dominó, podríamos decir que es un juego cuyas fichas se encuentran divididas en dos partes, las cuales señalan mediante incisiones o muescas, un número concreto entre los posibles palos o números admitidos. Estas fichas contendrán en su totalidad todas las combinaciones posibles de estos palos, comenzando por la ausencia de puntos o palo de blancos (González Sanz, 2010:24).
La Real Academia de la Lengua Española en su primera acepción define este juego como aquel que “se hace con 28 fichas rectangulares divididas en dos cuadrados, cada uno de los cuales lleva marcados de uno a seis puntos, o no lleva ninguno. Cada jugador pone por turno una ficha que tenga número igual en uno de sus cuadrados al de cualquiera de los dos que están en los extremos de la línea de las ya jugadas, y gana quien primero coloca todas las suyas o quien se queda con menos puntos, si se cierra el juego”. De este modo, la RAE acota algo más el término acercándose a lo que conocemos como dominó occidental.
Según la página http://www.domino-en-linea.com “los dominós son de simples bloques de construcción que pueden armarse de innombrables maneras con el fin de crear una gran variedad de juegos. Es un juego que exige mucha capacidad y de estrategia”.

****************************************************************************************
Los dominós serían partes de juego de origen chino. 
Se habrían datado los dominós de origen más antiguo, efectivamente se habría encontrado un juego de dominó en la tumba de Touthankamon en Egipto.

Los dominós nacieron de la derivación del juego de dados indio, conocido en Europa bajo el dado a seis-cara, los chinos modificaron este dado en parte plana reversible representando puntos, de 1 en 6 puntos. En Europa sería apareció una cara suplementaria, el blanco.
Partes fabricadas al origen en marfil.
La palabra “dominó” sería la causa, dut a una semejanza entre las partes del juego de dominó y la ropa de las religiosas las de Dominica (blanco cubierto de un cabo negro).
Los rastros más antiguos del juego de dominó que se puede encontrado en Europa datarían del siglo XVIII. Él aparecido en primer lugar en Italia antes de esto propagar al resto de Europa, que se han convertido en uno de los juegos más populares en las publicidades y la familia del alta.



Historia y origen de los dominós:
Lugar que trata de lorigine de los dominós, de China a través de Europa.

La invención del juego de dominó:
Un debate de la invención de los dominós en China, escrita porStewart Culin en 1893.
Según José Luis González San en El arte del dominó: teoría y práctica (2000:22) este juego parece ser muy antiguo o al menos juegos muy similares a este. Algunos historiadores creen que su origen es chino, ya que estos jugaban a algo parecido con impresiones en piedra y que llegó a Europa a través de comerciantes y mercaderes. Entre estos se nombra a Marco Polo. El juego viajó hasta Europa, concretamente hasta la  península itálica.
Martin Gardner, investigador, cuenta en su libro Circo Matemático que en los dominós chinos llamados Kwat p’ai no existen piezas con caras en blanco. En los dominós chinos cada pieza tiene un nombre pintoresco, por ejemplo, el 1-1 es la tierra, el 5-5 es la flor de ciruelo, el 6-5 es la cabeza de tigre…
Los dominós chinos suelen ser de cartón y se manejan como naipes. Al igual que en Occidente existen diversos juegos. Games of the Orient de Stewart Culin, obra de 1895 y reimpresa en 1958 por Charles Tuttle es una buena referencia.
Otras referencias atribuyen el juego a árabes o egipcios
 
Bibliografía
Lugo, M. (2008): Dominó competitivo. Bloomington: Author House
González Sanz, J.L. (2000): El arte del dominó: teoría y práctica. Barcelona: Paiditrobo.
http://www.librosmaravillosos.com/circomatematico/capitulo12.html Escrito por Patricio Barros.
http://www.rae.es
http://www.domino-en-linea.com/historia-del-domino.html
Enlace con mucha bibliografía sobre dominó.
 

\subsubsection{Reglas básicas}
\subsubsection{Estructura de una partida simple}
\subsubsection{El dominó es un juego de señores}
\subsubsection{Juego por parejas}
\subsubsection{Técnicas avanzadas}


\subsection{Inteligencia artificial}

\subsubsection{Sistemas expertos}
\subsubsection{Otros}


\chapter{Planificación}
% -*-cap2.tex-*-
% Este fichero es parte de la plantilla LaTeX para
% la realización de Proyectos Final de Carrera, protejido
% bajo los términos de la licencia GFDL.
% Para más información, la licencia completa viene incluida en el
% fichero fdl-1.3.tex

% Copyright (C) 2009 Pablo Recio Quijano 

\section{Planificación}

Para el desarrollo de \textbf{Dominous} se decidió utilizar el modelo evolutivo iterativo incremental para el ciclo de
vida del proyecto. La decisión fue tomada ya que, a pesar de tener acotado el ámbito y los requisitos del programa,
la funcionalidad concreta de cada uno de los apartados se desconocía en un principio.

El modelo iterativo incremental es un modelo de tipo evolutivo que está basado en varios ciclos cascada realimentados aplicados
repetidamente, con una filosofía iterativa. Como se comenta en CITE, al final de cada iteración se le
realiza una entrega al cliente final; en este caso, el cliente ha sido el tutor del proyecto, que a cada iteración iba
dictando las líneas maestras generales a tomar a cada nuevo paso.

Las ventajas de utilizar un modelo iterativo incremental son básicamente los siguientes:
\begin{enumerate}
    \item Construir un sistema pequeño es siempre menos costoso en términos de riesgo.
    \item Al ir desarrollando parte de las funcionalidades, es más fácil determinar si los requerimientos
            planeados para los niveles subsiguientes son correctos.
    \item Si se comete algún error grave, sólo la última iteración necesita ser descartada.
    \item Reduciendo el tiempo de desarrollo de un sistema (en este caso en incremento del sistema) decrecen las
            probabilidades que esos requerimientos de usuarios puedan cambiar durante el desarrollo.
    \item Los errores de desarrollo realizados en un incremento, pueden ser arreglados antes del comienzo del próximo incremento.
\end{enumerate}

El proyecto \textbf{Dominous} consta de tres subsistemas que son los que ocupan el grueso del desarrollo:
\begin{enumerate}
    \item El primero es el motor de la partida: Controla los jugadores, las fichas en la mesa, la partida, situaciones irregulares
            y cualquier otro elemento referente únicamente al ámbito del dominó.
    \item Por otro lado está el motor gráfico, que será la interfaz entre la partida y el usuario, permitiendo el movimiento
            fluído por las diferentes secciones del programa e interactuando de forma directa con el motor de la partida,
            mostrando las fichas actuales y habilitando la interacción del jugador con el mundo.
    \item Y por último tenemos el motor de Inteligencia Artificial. Este motor será el que alimente la inteligencia
            y las acciones y decisiones de los jugadores controlados por el ordenador.
\end{enumerate}

\subsection{Incrementos realizados}

A continuación se citan los diferentes incrementos que se han ido realizando en el desarrollo del proyecto.

\subsubsection{Preliminares}

El primer paso a la hora de enfrentarse a un proyecto es decidir las herramientas que vamos a utilizar. En principio se
pensó utilizar el lenguage C++ por dos sencillas razones:
\begin{enumerate}
    \item Por un lado es un lenguaje que hemos aprendido en la carrera, se ha utilizado en varias asignaturas de
            diferentes ramas, con lo cual la comodidad y familiaridad que podemos tener a la hora de programar
            es un punto importante a tener en cuenta.
    \item Tampoco podemos olvidar que, al ser un lenguaje compilado, la velocidad de ejecución que se consigue
            es interesante, y mucho más tratándose de temas como la inteligencia artificial (donde puede ser
            necesario un uso intensivo de los recursos del sistema) o el desarrollo de videojuegos (en el que
            la potencia del ordenador repercute en una mejor experiencia del usuario)
\end{enumerate}
    


\chapter{Análisis}
% ANÁLISIS

\section{Toma de requisitos}

En el desarrollo de esta aplicación la toma de requisitos se hizo mediante reuniones con el director del proyecto,
que realizaba el papel de cliente potencial de la misma. Después de varias reuniones se obtuvo el listado de requisitos
que se muestra en las siguientes secciones:

\subsection{Requisitos de interfaces externas}
\subsection{Requisitos funcionales}
\subsection{Requisitos de rendimiento}
\subsection{Requisitos del sistema software}

La aplicación debe cumplir con los siguientes requisitos de sistema:
\begin{itemize}
    \item La aplicación debe ejecutarse de forma multiplafatorma, incluyendo como mínimo los sistemas operativos:
        \begin{itemize}
            \item En Microsoft Windows --- Realizándose las pruebas en la versión Windows 7 con las últimas actualizaciones
            \item En sistemas GNU/Linux --- Utilizando la distribución Ubuntu en su versión 10.04 con su instalación
                    por defecto y con todas las actualizaciones del sistema
        \end{itemize}
    \item Lorem ipsum
\end{itemize}


\chapter{Diseño}
% -*-cap2.tex-*-
% Este fichero es parte de la plantilla LaTeX para
% la realización de Proyectos Final de Carrera, protegido
% bajo los términos de la licencia GFDL.
% Para más información, la licencia completa viene incluida en el
% fichero fdl-1.3.tex

% Copyright (C) 2009 Pablo Recio Quijano 

\section{Introducción}

El primer paso que vamos a dar para describir el diseño de la aplicación será describir los requisitos del
sistema necesarios para poder ejecutar la aplicación, y posteriormente comentaremos las herramientas que se
van a usar para el desarrollo de la aplicación.

\section{Definición de los requisitos del sistema}

Los requisitos hardware y software necesarios para poder ejecutar con soltura la aplicación son los siguientes:

\begin{itemize}
	\item Sistema operativo Microsoft Windows en su versión Windows 7, o sistemas basados en GNU/Linux tales como
		la distribución Ubuntu en su versión 10.04.
	\item Últimas actualizaciones del sistema instaladas, drivers y demás aplicaciones propias del sistema
		configuradas correctamente.
	\item Procesador igual o superior a 1,6 GHz.
	\item Memoria RAM igual o superior a 512 MB
	\item Tarjeta gráfica con aceleración 3D con un mínimo de 128 MB
	\item En versiones basadas en Linux, se requieren las siguientes librerías:
		\begin{itemize}
			\item python-pygame
			\item python-setuptools
			\item PyOpenGL
			\item PyOpenGL-accelerate
		\end{itemize}
	\item Por otro lado, si el sistema está basado en Microsoft Windows, necesitaremos:
		\begin{itemize}
			\item Pygame
			\item Python OpenGL and Gloss
		\end{itemize}
\end{itemize}

\section{Herramientas utilizadas}

\subsection{Librería gráfica}

Para todo el tema del aspecto gráfico, mi tutor me recomendó que utilizara las librerías SDL\footnote{Simple
Directmedia Layer}, ya que son unas librerías orientadas al desarrollo de videojuegos con varias particularidades:
\begin{itemize}
    \item Son completas, ya que permiten gestionar operaciones de dibujo en dos dimensiones, efectos de
            sonido y música, carga y gestión de imágenes, subsistemas de control de métodos de entrada,
            etcétera, por lo que contamos con una solución global para desarrollar videojuegos.
    \item Están programas en C, por lo que se puede esperar un buen rendimiento de las librerías en
            diferentes entornos.
    \item Multiplataforma: es compatible oficialmente con los sistemas Microsoft Windows, GNU/Linux,
            Mac OS y QNX, además de otras arquitecturas y sistemas menos comunes como Sega Dreamcast, Sony PSP,
            WebOS, Google Android o Symbian entre otros.
    \item Tampoco hay que mantener al margen la característica de que cuenta con wrappers a otros lenguajes
            de programación como entre los que se encuentran C++, Ada, C\#, BASIC, Erlang, Lua, Java o Python, por
            lo que nos da bastante libertad para elegir un lenguaje de programación principal
    \item Publicado bajo licencia LGPL, con todas las ventajas que conlleva.
    \item Y por último no hay que menospreciar que mi tutor emplea SDL a la hora de impartir la asignatura
            de diseño de videojuegos, y contar con esa base de conocimiento nos ayudará a desarrollar más rápidamente
            y solucionar antes nuestros posible problemas.
\end{itemize}

\begin{figure}[h]
  \label{logo-sdl}
  \begin{center}
    \includegraphics[scale=0.5]{SDL.png}
  \end{center}
  \caption{Logotipo de la librería Simple DirectMedia Layer}
\end{figure}

En este aspecto, la utilización de las librerías SDL estaba clara. Potencia, comodidad, multiplataforma y con la posibilidad
de utilizar diferentes lenguajes de programación.\\

\subsection{Lenguaje de programación}

Una vez que tocamos el tema de los lenguajes de programación, entra en escena la problemática sobre qué lenguaje
utilizar. En principio se pensó emplear el lenguaje C++ por dos sencillas razones:

\begin{enumerate}
    \item Por un lado es un lenguaje que hemos aprendido en la carrera, se ha utilizado en varias asignaturas de
            diferentes ramas, con lo cual la comodidad y familiaridad que podemos tener a la hora de programar
            es un punto importante a tener en cuenta.
    \item Tampoco podemos olvidar que, al ser un lenguaje compilado, la velocidad de ejecución que se consigue
            es interesante, y mucho más tratándose de temas como la inteligencia artificial (donde puede ser
            necesario un uso intensivo de los recursos del sistema) o el desarrollo de videojuegos (en el que
            la potencia del ordenador repercute en una mejor experiencia del usuario)
\end{enumerate}

Pero hay que detenerse un momento y pensar en la naturaleza del proyecto. Aunque el programa a desarrollar sea un
videojuego, no hay que olvidar que hay diferentes tipos de juegos, que pueden condicionar o influir en nuestra forma
de programarlo. En el caso del dominó, lo primero que debemos tener en cuenta es que el apartado gráfico no va a
requerir de una gran potencia o despliegue de efectos: el dominó es un juego pausado y a diferencia de otros
videojuegos lo importante en este caso es mostrar al usuario la información de la partida de una forma clara y sencilla,
para que el jugador evalúe las posibilidades de acción y actúe en consecuencia.\\

\begin{figure}[h]
  \label{logo-python}
  \begin{center}
    \includegraphics[scale=0.3]{python.png}
  \end{center}
  \caption{Logotipo de Python - Copyright Python Software Foundation}
\end{figure}


Si tenemos en cuenta estas circunstancias, existen otros lenguajes que también deben entrar en juego,
como por ejemplo \textbf{Python}. Buscando las diferencias, ventajas y desventajas de Python frente a C++,
obtenemos el siguiente listado:

\begin{enumerate}
    \item Python es un lenguaje de programación multiparadigma ya que soporta orientación a objetos,
            programación imperativa y, en menor medida, programación funcional.
    \item Al igual que C++ es multiplataforma, y está publicado con la licencia \textbf{Python Software Foundation
            License}, que es una licencia de software libre permisiva, compatible con la GPL.
    \item La sintaxis de Python es muy clara, simple, expresiva y legible, con lo cual los programas
            desarrollados bajo Python son más sencillos de entender \cite{Pilgrim:2004:DP:983200}.
    \item Python es un lenguaje interpretado, a diferencia de C++ que es compilado. Este aspecto podría suponer una
            desventaja ya que al ser interpretado puede resultar más lento, pero analicemos pausadamente estos
            factores:
        \begin{itemize}
            \item Como ya hemos comentado previamente, nuestra aplicación, a pesar de enmarcarse dentro de las
                    facciones de un videojuego, no requiere de grandes alardes de potencia gráfica como podría
                    suponerse, ya que es un tipo de juego pausado y donde cómo se muestra la información
                    es mucho más importante que la velocidad o los efectos de vídeo e imágenes.
            \item A pesar de ser interpretado, un gran conjunto de las funcionalidades de python --- como librerías o
                    funciones básicas del lenguaje --- están programadas internamente en C, así que podríamos
                    verlo como que estamos utilizando la comodidad de Python sobre la potencia de C, uniendo
                    lo mejor de ambos mundos.
        \end{itemize}
\end{enumerate}

\subsection{Diseño y estilo visual de interfaces}

El diseño visual que se ha desarrollado para la interfaz se ha creado desde cero, buscando los siguientes objetivos:
\begin{itemize}
    \item Líneas sencillas, minimalistas, sin recargar innecesariamente la pantalla
    \item Botones grandes, para que sea fácil de utilizar por usuarios de edad avanzada.
    \item Textos con un punto de letra elevado, facilitando la rápida lectura y legibilidad del texto.
\end{itemize}

\begin{figure}[h]
  \label{interfaz}
  \begin{center}
    \includegraphics[scale=0.5]{ui.png}
  \end{center}
  \caption{Diferentes elementos utilizados en la interfaz}
\end{figure}

\subsection{Documentación del código}

Por otro lado, para gestionar toda la documentación del proyecto se decidió utilizar las siguientes herramientas:
\begin{itemize}
    \item \LaTeX\ para escribir la memoria, ya que es una forma robusta y fiable de escribir una memoria para
            un Proyecto Fin de Carrera, descartándose otras posibles opciones por no ser adecuadas para la escritura
            de un documento de estas características. Para facilitar la compilación dispone de la herramienta
            GNU Make~\cite{pdf:make}.
    \item Doxygen para la documentación del código fuente, porque además de documentar de manera sencilla y fácil
            de leer el mismo código fuente, genera una documentación en diferentes formatos. Además,
        \begin{itemize}
            \item Doxygen funciona con lenguajes como C++, C, Java, Objective-C, Python, Fortran, VHDL, PHP o C\#
                    (entre otros), por lo que se puede acomodar a nuestras necesidades. Incluso existe una
                    herramienta llamada \textbf{Doxypy} que nos permite reutilizar los comentarios \emph{tipo Python}
                    y adaptarlos a Doxygen, con lo cual ahorramos trabajo y cumplimos con la normativa
                    de código Python.
        \end{itemize}
\end{itemize}

\subsection{Sistema de control de versiones}

El código del proyecto Dominous, está alojado por completo dentro del sistema que proporciona Rediris, que básicamente
consiste en un entorno completo basado en Subversion -- SVN. \\ 

Subversion permite llevar un control exahustivo de todos los ficheros e iteraciones de código que se realizan en él,
permitiendo volver a versiones anteriores de código, comprobar diferencias entre versiones o ficheros y cualquier otra
operación propia de un sistema de control de versiones. \\

Se evaluaron otros sistemas de control de versiones distribuidos como GIT, Bazaar o Mercurial, pero se desecharon
básicamente porque, por un lado, este proyecto cuenta únicamente con un desarrollador, y multitud de ventajas que
ofrecen los sistemas de control de versiones distribuidos dejan de tener sentido, si tenemos en cuenta esta circunstancia
del proyecto, y por otro lado la integración de SVN con Rediris (con las ventajas de visualización de código y versiones
que ofrece ViewCVS) decantaron la decisión sobre el lado de SVN.

\section{Interfaz gráfica}

Tomando los resultados obtenidos en la fase de análisis es necesario diseñar una interfaz gráfica amigable
para el usuario y desde la cual se pueda interactuar con la aplicación. Para el diseño de las interfaces
se intentará en todo momento que sean usables además de intentar conseguir que el usuario no pueda
introducir datos erróneas para que no produzca comportamientos anómalos. \\

\begin{figure}[h]
  \label{fig:pantallas_interfaz}
  \begin{center}
    \includegraphics[scale=0.25]{interfaz.png}
  \end{center}
  \caption{Varias pantallas con la interfaz de Dominous}
\end{figure}

\subsection{Diagrama de interacción entre interfaces gráficas}

En el siguiente diagrama~\ref{fig:diagramainteraccioninterfaces} podemos observar la interacción entre
las distintas interfaces gráficas desarrolladas para la aplicación. \\

\begin{figure}[h]
  \label{fig:diagramainteraccioninterfaces}
  \begin{center}
    \includegraphics[scale=0.15]{diagrama_interfaces.png}
  \end{center}
  \caption{Diagrama de interacción entre interfaces}
\end{figure}

\section{Diagrama Entidad -- Relación}

La aplicación \textbf{Dominous} realiza un almacenamiento limitado de información, por lo que no se estima necesario
realizar un diagrama Entidad -- Relación con este fin.

\section{Diagrama de clases de diseño}

A continuación se muestra el diagrama de clases de diseño para \textbf{Dominous}.

\begin{figure}[h]
  \label{diagrama_clases_diseno}
  \begin{center}
    \includegraphics[scale=0.23]{diagrama_clases_diseno.png}
  \end{center}
  \caption{Diagrama de diseño del sistema}
\end{figure}


\section{Análisis de las principales clases de la aplicación}

En este apartado realizaremos un repaso a las principales clases que intervienen en el diseño de \textbf{Dominous},
definiendo los métodos y atributos que se requieren para el correcto desarrollo de la aplicación y comentando
todos los apartados cuya funcionalidad merezca ser destacada.

\subsection{Clase dominous}

La clase dominous genera el objeto principal que soporta el peso principal de toda la aplicación. Al inicio de la
ejecución se crea un objeto de esta clase; este objeto va creando --- según las necesidades del flujo que tome
la ejecución de la misma --- los diferentes objetos, controlando y dando paso a la interfaz activa, llamando a
las principales funciones y permitiendo la captura de los eventos de entrada. \\

Al igual que la gran mayoría de videojuegos, para el desarrollo de Dominous se han utilizado dos métodos principales
que permiten controlar la aplicación en tiempo real. Estos dos métodos son:
\begin{description}
    \item[draw] El primer método se encarga de dibujar en pantalla todos los elementos de la interfaz. El funcionamiento
        básico es recorrer todos los objetos activos y pintarlos según la posición, rotación y escala que tengan en
        ese preciso momento. 
    \item[update] El segundo método se ocupa de realizar los cálculos para el movimiento de los elementos gráficos. 
        En cada iteración cambiará el estado de los objetos activos, modificando la imagen o los valores posición,
        rotación y/o escala, estudiando las posibles colisiones, eventos y cualquier otra circunstancia que cambie
        el estado de la aplicación, para que en el siguiente ciclo se dibujen los elementos en su posición correcta.
\end{description}

Una vez que este objeto principal toma el control de la aplicación, su función es llamar a los métodos \comando{draw}
y \comando{update} del objeto encargado de la interfaz que estemos mostrando actualmente, de forma alternativa.

Este juego de llamadas a ambos métodos se realizará hasta un máximo de 60 veces por segundo, dependiendo del rendimiento
que se obtenga de los recursos del sistema. De esta forma conseguiremos un máximo de 60 frames o cuadros por segundo,
que serán los encargados de generar la sensación de movimiento.

Como vemos, el objeto de la clase dominous no realiza ningún dibujado en pantalla ni cálculos, sino que su único
cometido es redireccionar el flujo de la aplicación al objeto correspondiente.

\subsection{Clase dominoes\_game}

La labor de esta clase es llevar el control de una partida completa de dominó, siguiendo las reglas del \textbf{dominó
internacional}. Dependiendo de la configuración de la partida, creará los jugadores según la selección que se haya
realizado en la pantalla de selección de personajes y comenzará con el bucle principal: \\

Repartirá las fichas entre los diferentes participantes del juego e irá pidiendo fichas a los jugadores de forma
consecutiva, hasta que se de alguna circunstancia de finalización de la mano. \\

Este bucle se repetirá hasta que alguno de los dos equipos alcance o supere los 200 puntos, momento en el cual la partida
se dará por finalizada. \\

Esta clase también mantendrá activo y actualizado un fichero de log con toda la información generada en la partida. Esta
información se utilizará más tarde en el apartado de inteligencia artificial, y almacenará los datos utilizando
la siguiente estructura: \\

Por un lado tenemos la información global de la partida: 

\begin{description}
    \item[date] Fecha de desarrollo de la partida.
    \item[place] Lugar donde tuvo lugar la partida.
    \item[max\_points] Límite superior de puntos necesarios para ganar el encuentro. Como se utilizan las reglas del
        \textbf{dominó internacional}, esta cota estará situada en los doscientos puntos.
    \item[player1] Datos del jugador 1.
    \item[player2] Datos del jugador 2.
    \item[player3] Datos del jugador 3.
    \item[player4] Datos del jugador 4.
    \item[team1] Pareja que forma el equipo 1.
    \item[team2] Pareja que forma el equipo 2.
    \item[points\_team1] Puntos actuales que tiene el equipo 1.
    \item[points\_team2] Puntos actuales que tiene el equipo 2.
    \item[winner\_team] Pareja que finalmente ha sido ganadora de la partida.
\end{description}

Y por otro lado tenemos información relativa a la mano que se está desarrollando actualmente. Esta información se repite
por cada movimiento que se realice en la partida, y se agrupa por manos:

\begin{description}
    \item[player] Jugador que ha realizado el movimiento.
    \item[tile] Ficha que ha colocado.
    \item[side] Lado del tablero por donde ha colocado la ficha.
    \item[left] Cifra, del cero al siete, que se tiene como resultante del movimiento por el lado izquierdo del tablero.
    \item[right] Cifra, del cero al siete, que se tiene como resultante del movimiento por el lado derecho del tablero.
    \item[mtime] Tiempo (en segundos) que ha estado pensando el jugador antes de colocar la ficha.
\end{description}


\section{Sistemas expertos}




\chapter{Implementación}
% -*-cap2.tex-*-
% Este fichero es parte de la plantilla LaTeX para
% la realización de Proyectos Final de Carrera, protegido
% bajo los términos de la licencia GFDL.
% Para más información, la licencia completa viene incluida en el
% fichero fdl-1.3.tex

% Copyright (C) 2009 Pablo Recio Quijano 

\section{Implementación}

Una vez terminadas las dos fases previas --- análisis y diseño --- es hora de enfrentarnos a la codificación e implementación
en sí de la aplicación. Este apartado es largo y entretenido, y va descubriendo a cada momento nuevas problemáticas que
no se han tenido en cuenta antes y es necesario resolver, pero también es apasionante, porque vamos viendo de forma
empírica cómo nuestra aplicación va tomando forma y se va haciendo tangible a cada momento que avanza. \\

El desarrollo de Dominous ha presentado en diferentes momentos varios problemas y dificultades que creemos interesante destacar,
y que se pasamos a comentar a continuación.

\subsection{Entorno gráfico}

El entorno gráfico se puede dividir en dos apartados de diferente complejidad: por un lado tenemos la gestión de menús,
opciones, pantallas y secciones de la aplicación, y por otro lado tenemos la gestión de una partida de dominó, con todas
sus interacciones, animaciones y movimientos de fichas, eventos y demás.

\subsubsection{Sistema de secciones}

Todo el sistema de secciones lo gobierna el objeto principal de la aplicación, llamado \textbf{dominous}. Este objeto posee
diferentes objetos como atributos, y estos atributos son las diferentes secciones de la aplicación. Cuando queremos
que el flujo de la aplicación pase de una sección a otra no tenemos más que llamarla; al llamarla estaremos ejecutando
el siguiente fragmento

\begin{lstlisting} [caption={[]Ejemplo de listado}, language=Python]

def goto_tutorial(self):
    self.flow.stop()
    self.flow = self.tutorial
    self.flow.start()


\end{lstlisting}

\subsubsection{Gestión de partida}

Lorem ipsum

\subsection{Inteligencia Artificial}

Dolor sit amet


\chapter{Pruebas}
% -*-cap2.tex-*-
% Este fichero es parte de la plantilla LaTeX para
% la realización de Proyectos Final de Carrera, protegido
% bajo los términos de la licencia GFDL.
% Para más información, la licencia completa viene incluida en el
% fichero fdl-1.3.tex

% Copyright (C) 2009 Pablo Recio Quijano 

\section{Pruebas}

La fase de prueba es de las más importantes de un proyecto software~\cite{art}. El objetivo de este paso es, como razón
última, la verificación de que el proyecto cumple con todos los requisitos iniciales que se plantearon al comienzo 
del desarrollo. Según la metodología clásica de desarrollo de pruebas existen diferentes enfoques a la hora de realizar
este tipo de pruebas de software, siendo todos complementarios~\cite{beizer_software_1990}, pero para el caso
concreto que tenemos entre manos debemos tener en cuenta la naturaleza del proyecto. \\

Los videojuegos presentan dos facetas distintas que deben ser abordadas de diferentes maneras cuando se realiza la fase
de pruebas:
\begin{itemize}
    \item Por un lado tenemos las pruebas clásicas que se realizan a cualquier desarrollo de software, como pueden
            ser las pruebas unitarias o de integración, destinadas a verificar el correcto funcionamiento del
            código.
    \item Y por otro lado, al requerir de una interacción directa con el usuario (y estar destinado a divertir,
            entretener y proporcionar un rato ameno al mismo) se deben realizar varios tipos de pruebas orientadas
            a comprobar que se cumplen requisitos más abstractos como puede ser el testeo o análisis de la jugabilidad
            o usabilidad de la aplicación, interfaces, medir el entretenimiento que proporciona la aplicación (relacionado
            directamente con el desarrollo de la inteligencia artificial de los contrincantes, entre otros)
\end{itemize}

El primer conjunto de pruebas se pueden realizar con comprobaciones de código, compilación (o en este caso concreto,
interpretación) del mismo y diferentes métodos, pero el segundo conjunto de pruebas es necesario que se desarrollen
con diferentes tipos de usuarios reales manejando la aplicación y realizando cuestionarios que nos ayuden a valorar
el éxito o fracaso de estos apartados. \\

Para los apartados de pruebas unitarias y de integración se va a trabajar principalmente con los tres módulos que
forman el núcleo fuerte de la aplicación (y sobre las que cae el peso de la misma):

\begin{itemize}
    \item Sistema de gestión de partida de dominó
    \item Sistema de inteligencia artificial
    \item Motor gráfico de la aplicación
\end{itemize}

\subsection{Pruebas unitarias}

Durante la fase de implementación se fueron realizando pruebas unitarias no automatizadas de cada conjunto o subconjunto
de módulos que se iban desarrollando, evitando así encontrar errores en las pruebas de integración que no sean propiamente
de integración sino de errores en la codificación de los diferentes módulos. \\

\subsubsection{Sistema de gestión de partida de dominó}

En este módulo las pruebas unitarias están claras: el módulo debe velar por el correcto cumplimiento de todas y cada
una de las reglas del dominó internacional. Para ello debe vigilarse cada movimiento de los jugadores, repasando
el estado del juego y las diferentes posibilidades que tiene el jugador, entrándose en modo error cuando se realiza
una acción ilegal. \\

En este momento del desarrollo se tuvo que tomar una decisión respecto a cómo contemplar los errores o incumplimiento de
normas que puedan producirse (de forma intencionada o no) por parte de los jugadores:

\begin{itemize}
    \item Una opción era permitir que los jugadores incumplan las normas y actuar en consecuencia contra el jugador.
            En ningún momento se puede permitir que los jugadores coloquen fichas en lugares donde está prohibido
            ese movimiento, por lo que, de todas las normas que posee el dominó las únicas candidatas a entrar en este
            grupo son las que hacen del dominó un \textbf{juego de señores}, es decir, todas aquellas que están destinadas
            a dotar de información a los demás jugadores de forma obligatoria (como puede ser la falta de un palo mediante
            acortar el tiempo que transcurre en su turno). \\

            Para esta opción teníamos nuevamente otras dos opciones:
            \begin{itemize}
                \item Podemos permitir y \emph{mirar hacia otro lado}, permitiendo que los jugadores engañen de forma
                        clara a los demás jugadores, incluyendo compañeros, o
                \item Podemos actuar como jueces y penalizar a los jugadores que cometan este tipo de faltas.
            \end{itemize}
    \item La otra opción es no permitir este tipo de acciones, volviendo hacia atrás en la acción del jugador y
            pidiendo de nuevo que actúe, hasta que la acción satisfaga las reglas del juego del dominó.
\end{itemize}

Finalmente se decidió que, dado que esta aplicación tiene como requisito el favorecer el aprendizaje del juego del dominó,
no existe cabida alguna a jugadas deshonrosas o que puedan llevar a equivocación al usuario que desea aprender a jugar. \\

Por lo tanto, cuando un jugador intenta realizar una acción equivocada, se deshace esa acción y se le pide de nuevo que
mueva ficha hasta que ese movimiento sea correcto, momento en el cual se continúa con naturalidad la partida.



\subsection{Pruebas de integración}

Según se iban desarrollando módulos y éstos cumplían las pruebas unitarias desarrolladas para estos apartados, era necesario
integrar los diferentes módulos para corroborar y contrastar el correcto funcionamiento de la conjunción de ambos módulos.



\chapter{Conclusiones}
% -*-cap2.tex-*-
% Este fichero es parte de la plantilla LaTeX para
% la realización de Proyectos Final de Carrera, protegido
% bajo los términos de la licencia GFDL.
% Para más información, la licencia completa viene incluida en el
% fichero fdl-1.3.tex

% Copyright (C) 2009 Pablo Recio Quijano 

\section{Resultados obtenidos}

Primer simulador de dominó libre.

Aplicación de lo aprendido en la carrera. En especial, ampliación de los conceptos de IA estudiados ¿Llegaste a hacer la asignatura?.

Aprendizaje del juego de dominó.

Primer proyecto que realizo de manera íntegra.

\section{Trabajos futuros}

Juego en red.

Servidor web para compartir sistemas expertos.

Analizador de partidas con aprendizaje automático.



%\backmatter % Apéndices, bibliografia ...

\appendix

\chapter{Manual de instalación}
% -*-cap2.tex-*-
% Este fichero es parte de la plantilla LaTeX para
% la realización de Proyectos Final de Carrera, protegido
% bajo los términos de la licencia GFDL.
% Para más información, la licencia completa viene incluida en el
% fichero fdl-1.3.tex

% Copyright (C) 2009 Pablo Recio Quijano 

\section{Obtener Dominous}

Dominous se puede descargar desde varias fuentes principales. Dado que el lenguaje es interpretado y no compilado, el código
es válido tanto para sistemas basados en GNU/Linux como para Windows 7. \\

La descarga se puede hacer desde la web oficial~\cite{website:dominous} o desde los repositorios alojados en la 
forja de Rediris , en el que además del código fuente se cuenta con
herramientas como foros de consulta, descarga de manuales, bug tracker y otros.

\begin{figure}[h]
  \begin{center}
    \includegraphics[scale=0.65]{pagina_oficial_dominous.png}
  \end{center}
  \caption{Página oficial de Dominous}
  \label{fig:pagina_oficial_dominous}
\end{figure}

\section{Instalación y ejecución en GNU/Linux}

La instalación bajo sistemas GNU/Linux requiere de cierto número de dependencias, que pueden ser instaladas cómodamente
con el siguiente comando:

\begin{lstlisting} [caption={Instalación de dependencias}, language=Bash, numbers=left]
user@machine:~$ sudo aptitude install python-pygame
                    python-setuptools && sudo easy_install
                    PyOpenGL PyOpenGL-accelerate
\end{lstlisting}

Como vemos, las dependencias que requiere la aplicación son las siguientes:

\begin{itemize}
    \item python-pygame
    \item python-setuptools
    \item PyOpenGL
    \item PyOpenGL-accelerate
\end{itemize}

Por último, para ejecutar y disfrutar de Dominous, solamente debemos acceder al directorio y ejecutar:

\begin{lstlisting} [language=Bash, numbers=left]
user@machine:~$ python dominous.py
\end{lstlisting}

\section{Instalación y ejecución en Windows}

Por otra parte, si deseamos instalar la aplicación en un sistema Windows, debemos ejecutar los siguientes pasos:

\begin{itemize}
    \item Obtener e instalar Python --- Para ello nos dirigimos a la página oficial de Python,
        descargamos la última versión estable (que en el momento de escribir este documento era la 2.7.2) y seguimos los pasos
        de instalación.
    \item Obtener e instalar PyOpenGL --- descargamos la última versión desde la página oficial de
        PyOpenGL y seguimos los pasos necesarios hasta instalarla
        correctamente en nuestro sistema.    
\end{itemize}

Para finalizar, nos dirigimos a la carpeta de Dominous y hacemos doble click sobre el icono de \comando{dominous.py}.


\chapter{Manual de usuario}
% -*-cap2.tex-*-
% Este fichero es parte de la plantilla LaTeX para
% la realización de Proyectos Final de Carrera, protegido
% bajo los términos de la licencia GFDL.
% Para más información, la licencia completa viene incluida en el
% fichero fdl-1.3.tex

% Copyright (C) 2009 Pablo Recio Quijano 

% MANUAL DE USUARIO

% FIXME
% subir al apartado de ficehros de RedIRIS los manuales

\section{Ejecución}

En este capítulo veremos el funcionamiento de la aplicación desde el punto de vista de un usuario final que ejecutará el
videojuego con la intención de disfrutar de las opciones que ofrece. \\



as Si el programa está instalado, se ejecuta con XXX

\section{Menú principal}




\chapter{Difusión}
% -*-cap2.tex-*-
% Este fichero es parte de la plantilla LaTeX para
% la realización de Proyectos Final de Carrera, protegido
% bajo los términos de la licencia GFDL.
% Para más información, la licencia completa viene incluida en el
% fichero fdl-1.3.tex

% Copyright (C) 2009 Pablo Recio Quijano 

\section{Difusión}

% FIXME TODO

Página web de RedIRIS con el Doxygen

% Subir al apartado de ficehros de RedIRIS los manuales

Inclusión en Guadalinex

Premio CUSL5-UCA: Accésit al mejor proyecto de innovación




%\addcontentsline{toc}{chapter}{Software usado}
%\chapter*{Software utilizado}
%\input{programas.tex}

%\addcontentsline{toc}{chapter}{Instalación de \LaTeX}
%\chapter*{Instalación de \LaTeX}
%\input{instalacion.tex}

\clearpage
\addcontentsline{toc}{chapter}{Bibliografia y referencias}
%%\bibliographystyle{plain}
\bibliographystyle{spanish}
\bibliography{bibliografia}

\input{fdl-1.3.tex}

\end{document}
