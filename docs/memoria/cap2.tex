% -*-cap2.tex-*-
% Este fichero es parte de la plantilla LaTeX para
% la realización de Proyectos Final de Carrera, protejido
% bajo los términos de la licencia GFDL.
% Para más información, la licencia completa viene incluida en el
% fichero fdl-1.3.tex

% Copyright (C) 2009 Pablo Recio Quijano 

\section{Conceptos básicos}

\subsection{El dominó}

\subsubsection{Historia del dominó}

El dominó, deporte y pasatiempo a un tiempo, es un juego que ha ganado adeptos a lo
largo de la historia y del tiempo. Según explica Miguel Lugo en su libro Dominó Competitivo
éste “es entretenido y fácil de aprender. Ya desde pequeños comenzamos a jugar al dominó con
frutas o con animales en lugar de hacerlo con puntos”. Además, éste ha ganado popularidad
durante los últimos años hasta el punto de televisar torneos en Europa, Norteamérica y Latinoamérica.
“En 2001 la Federación Internacional de Dominó (con sede en Barcelona, España) celebró el primer Congreso Internacional. El año siguiente se llevó a cabo el primer Campeonato del Mundo de Dominó en La Habana (Cuba). El ‘Mundial’ continúa celebrándose anualmente a partir de este punto, en España, México, Venezuela y EEUU” (Lugo, 2008: 1).

Así pues Lugo señala que el dominó nunca antes ha estado tan reconocido en todo el mundo como ahora e incluso afirma que se puede jugar por internet.

La página web http://www.domino-en-linea.com recoge que en la actualidad el dominó se juegan en todo el mundo aunque resalta que es “especialmente popular en América Latina, donde los dominós se consideran como el juego nacional de numerosos países del Caribe”. Así, también menciona los torneos anuales y los clubes locales de dominó.

“El origen del dominó parece ser muy antiguo, al menos en lo que se refiere a juegos similares y quizá pretéritos al actual” (González Sanz, 2010:22)

Cuenta Gonzalez Sanz en su libro El arte del dominó: teoría y práctica que algunos historiadores creen que puede tener origen chino, ya que éstos jugaban a un juego parecido con impresiones en piedra, y que pudo llegar a Europa a través de mercaderes y viajeros, entre los que se cita al célebre Marco Polo, los cuales y fruto de los intercambios culturales de la época, trajeron el dominó a este lado del mundo, más en concreto a la Península Itálica, primer lugar de Europa donde se ha datado la práctica de este juego.

Respecto al juego chino, Martin Gardner experto en juegos que colaboró más de 20 años en la sección “Juegos Matemáticos” de la revista Scientific American, comenta en su libro Circo Matemático que en los dominós chinos, llamado kwat p’ai, no existen piezas con caras en blanco. Éstos contienen todas las combinaciones por pares desde el (1-1)  hasta el (6-6), donde tres de los seis puntos de cara  son también rojos. Los dominós coreanos son idénticos, con la única particularidad de que en el as, el punto es mayor que en las demás piezas. En los dominós chinos, cada pieza tiene un nombre pintoresco: el (6-6) es el “cielo”; el (1-1) es “la tierra”, el (5-5) es la “flor del ciruelo”, el (6-5), “la cabeza de tigre”, etc. Los nombres de las piezas son iguales a los que reciben los 21 resultados posibles del lanzamiento de un par de dados.

Precisamente este autor Gardner explica que en la literatura occidental no hay referencias a este juego hasta mediados del siglo XVIII, en que empezaron a jugarse en Italia y Francia las primeras partidas. Desde ahí, el juego se extendió al resto del continente, y más tarde, a Inglaterra y América. En Occidente, la colección normal de piezas de dominó ha consistido siempre en 28 teselas o losetas formadas por dos cuadrados adyacentes, que contienen todos los posibles pares de dígitos, de 0 hasta 6. 

González Sanz (2010:22) señala que los dominós chinos “suelen ser en la actualidad de cartón, en vez de la madera, marfil, pasta, o ébano que es lo habitual en los occidentales, y se manejan como naipes”. Y al igual que ocurre en Europa y América, con estas piezas se realizan numerosos juegos. Respecto a los distintos juegos de dominó chinos y coreanos, este autor nos remite a Games of the Orient de Stewart Culin, obra de 1895 reimpresa en 1958 por Charles Tuttle como la mejor referencia. Asimismo apunta que no existe un dominó propio en Japón –frente al resto de países asiáticos- y dice que en este país se juega con el sistema occidental.

Sin embargo, y aunque por lo que señalan algunos autores el origen asiático del dominó es el más extendido, también existen otras versiones que atribuyen el invento del juego a árabes o egipcios, “sosteniendo que no hay pruebas para relacionar claramente el dominó europeo con el chino, pudiendo ser dos invenciones independientes y separadas en el tiempo” (2010:23). González también cuenta que se conocen otras versiones del juego como la esquimal o la coreana, con distinto número de fichas y palos al clásico.
 
TIPOLOGÍA DE DOMINÓS

Como hemos visto, a lo largo y ancho del mundo existen diferentes tipos de dominó de los que no siempre los autores coinciden con un solo origen. Aunque para catalogar tales juegos como dominó sí deben tener una serie de características en común.
Generalizando el concepto de dominó, podríamos decir que es un juego cuyas fichas se encuentran divididas en dos partes, las cuales señalan mediante incisiones o muescas, un número concreto entre los posibles palos o números admitidos. Estas fichas contendrán en su totalidad todas las combinaciones posibles de estos palos, comenzando por la ausencia de puntos o palo de blancos (González Sanz, 2010:24).
La Real Academia de la Lengua Española en su primera acepción define este juego como aquel que “se hace con 28 fichas rectangulares divididas en dos cuadrados, cada uno de los cuales lleva marcados de uno a seis puntos, o no lleva ninguno. Cada jugador pone por turno una ficha que tenga número igual en uno de sus cuadrados al de cualquiera de los dos que están en los extremos de la línea de las ya jugadas, y gana quien primero coloca todas las suyas o quien se queda con menos puntos, si se cierra el juego”. De este modo, la RAE acota algo más el término acercándose a lo que conocemos como dominó occidental.
Según la página http://www.domino-en-linea.com “los dominós son de simples bloques de construcción que pueden armarse de innombrables maneras con el fin de crear una gran variedad de juegos. Es un juego que exige mucha capacidad y de estrategia”.

****************************************************************************************
Los dominós serían partes de juego de origen chino. 
Se habrían datado los dominós de origen más antiguo, efectivamente se habría encontrado un juego de dominó en la tumba de Touthankamon en Egipto.

Los dominós nacieron de la derivación del juego de dados indio, conocido en Europa bajo el dado a seis-cara, los chinos modificaron este dado en parte plana reversible representando puntos, de 1 en 6 puntos. En Europa sería apareció una cara suplementaria, el blanco.
Partes fabricadas al origen en marfil.
La palabra “dominó” sería la causa, dut a una semejanza entre las partes del juego de dominó y la ropa de las religiosas las de Dominica (blanco cubierto de un cabo negro).
Los rastros más antiguos del juego de dominó que se puede encontrado en Europa datarían del siglo XVIII. Él aparecido en primer lugar en Italia antes de esto propagar al resto de Europa, que se han convertido en uno de los juegos más populares en las publicidades y la familia del alta.



Historia y origen de los dominós:
Lugar que trata de lorigine de los dominós, de China a través de Europa.

La invención del juego de dominó:
Un debate de la invención de los dominós en China, escrita porStewart Culin en 1893.
Según José Luis González San en El arte del dominó: teoría y práctica (2000:22) este juego parece ser muy antiguo o al menos juegos muy similares a este. Algunos historiadores creen que su origen es chino, ya que estos jugaban a algo parecido con impresiones en piedra y que llegó a Europa a través de comerciantes y mercaderes. Entre estos se nombra a Marco Polo. El juego viajó hasta Europa, concretamente hasta la  península itálica.
Martin Gardner, investigador, cuenta en su libro Circo Matemático que en los dominós chinos llamados Kwat p’ai no existen piezas con caras en blanco. En los dominós chinos cada pieza tiene un nombre pintoresco, por ejemplo, el 1-1 es la tierra, el 5-5 es la flor de ciruelo, el 6-5 es la cabeza de tigre…
Los dominós chinos suelen ser de cartón y se manejan como naipes. Al igual que en Occidente existen diversos juegos. Games of the Orient de Stewart Culin, obra de 1895 y reimpresa en 1958 por Charles Tuttle es una buena referencia.
Otras referencias atribuyen el juego a árabes o egipcios
 
Bibliografía
Lugo, M. (2008): Dominó competitivo. Bloomington: Author House
González Sanz, J.L. (2000): El arte del dominó: teoría y práctica. Barcelona: Paiditrobo.
http://www.librosmaravillosos.com/circomatematico/capitulo12.html Escrito por Patricio Barros.
http://www.rae.es
http://www.domino-en-linea.com/historia-del-domino.html
Enlace con mucha bibliografía sobre dominó.
 

\subsubsection{Reglas básicas}
\subsubsection{Estructura de una partida simple}
\subsubsection{El dominó es un juego de señores}
\subsubsection{Juego por parejas}
\subsubsection{Técnicas avanzadas}


\subsection{Inteligencia artificial}

\subsubsection{Sistemas expertos}
\subsubsection{Otros}
