% -*-cap2.tex-*-
% Este fichero es parte de la plantilla LaTeX para
% la realización de Proyectos Final de Carrera, protegido
% bajo los términos de la licencia GFDL.
% Para más información, la licencia completa viene incluida en el
% fichero fdl-1.3.tex

% Copyright (C) 2009 Pablo Recio Quijano 

\section{El dominó}

\subsection{Historia del dominó}

\subsection{Reglas básicas}
\subsection{Estructura de una partida simple}
\subsection{El dominó es un juego de señores}
\subsection{Juego por parejas}
\subsection{Técnicas avanzadas}

\section{Inteligencia artificial}

A la hora de afrontar un proyecto que simule cierto comportamiento \emph{humano}, debemos acercarnos a esa rama de la
Informática llamada Inteligencia Artificial, en busca de herramientas, técnicas y metodologías que nos ayuden a afrontar este
difícil problema, probablemente uno de los más complicados dentro de la Ingeniería Informática

\subsection{Sistemas expertos}

Para impregnar de inteligencia a los contrincantes de Dominous se utilizará lo que se suele llamar un \textbf{sistema experto}~\cite{Giarratano:1989:ESP:583478}.
Los sistemas expertos son una rama de la Inteligencia Artificial, que se basa en imitar los mecanismos y la forma de
pensar de un experto en cierta materia para resolver problemas de distinta índole.\\

Un Sistema Experto está conformado por:
\begin{description}
    \item[Base de conocimientos] Contiene conocimiento modelado extraído del diálogo con un experto.
    \item[Base de hechos (Memoria de trabajo)] Contiene los hechos sobre un problema que se ha descubierto durante el análisis.
    \item[Motor de inferencia] Modela el proceso de razonamiento humano.
    \item[Módulos de justificación] Explica el razonamiento utilizado por el sistema para llegar a una determinada conclusión.
    \item[Interfaz de usuario] Es la interacción entre el SE y el usuario, y se realiza mediante el lenguaje natural.
\end{description}

\subsection{Otros}
