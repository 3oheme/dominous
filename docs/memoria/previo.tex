% -*-previo.tex-*-
% Este fichero es parte de la plantilla LaTeX para
% la realización de Proyectos Final de Carrera, protegido
% bajo los términos de la licencia GFDL.
% Para más información, la licencia completa viene incluida en el
% fichero fdl-1.3.tex

% Copyright (C) 2009 Pablo Recio Quijano 

\section*{Agradecimientos}

Gracias a los que han estado conmigo desde siempre, a las nuevas incorporaciones y, sobre todo, a mi padre.

\cleardoublepage

\section*{Licencia} % Por ejemplo GFDL, aunque puede ser cualquiera

Este documento ha sido liberado bajo Licencia GFDL 1.3 (GNU Free
Documentation License). Se incluyen los términos de la licencia en
inglés al final del mismo.\\

Copyright (c) 2011 Ignacio Palomo Duarte.\\

Permission is granted to copy, distribute and/or modify this document under the
terms of the GNU Free Documentation License, Version 1.3 or any later version
published by the Free Software Foundation; with no Invariant Sections, no
Front-Cover Texts, and no Back-Cover Texts. A copy of the license is included in
the section entitled "GNU Free Documentation License".\\

\cleardoublepage

\section*{Notación y formato}

Cuando nos refiramos a un programa en concreto, utilizaremos la
notación: \programa{emacs}.\\

Cuando nos refiramos a un comando, o función de un lenguaje, usaremos
la notación: \comando{quicksort}.\\

Extractos de ficheros con texto plano o código aparecerán

\begin{lstlisting} [language=Python, numbers=left]
class nombre_de_la_regla:
    def __init__(self):
        # inicializamos los atributos que sean necesarios
        pass
    def go(self, left_tile, right_tile, board, tiles, log):
        return ficha, lado, tiempo_pensando
\end{lstlisting}
