% -*-cap2.tex-*-
% Este fichero es parte de la plantilla LaTeX para
% la realización de Proyectos Final de Carrera, protegido
% bajo los términos de la licencia GFDL.
% Para más información, la licencia completa viene incluida en el
% fichero fdl-1.3.tex

% Copyright (C) 2009 Pablo Recio Quijano 

\section{Resultados obtenidos}

Los resultados obtenidos gracias al desarrollo de esta aplicación han sido tremendamente satisfactorios, tanto en el
plano personal como en el profesional. \\

Como hitos conseguidos podemos destacar los siguientes:

\subsection{Primer simulador de dominó libre}

Sin querer resultar pretencioso --- y siempre según la investigación previa que hemos realizado --- Dominous es el primer
videojuego de dominó internacional bajo licencias libres, con todas las ventajas que ello conlleva:

\begin{itemize}
    \item Posibilidad de que pueda ser mejorado por la comunidad, añadiendo soluciones a errores de código o refactorizaciones,
        proporcionando nuevos jugadores, mejorando el sistema y motor de Inteligencia Artificial, añadiendo nuevas
        funcionalidades que se encuentran fuera de los requisitos iniciales del proyecto, portar Dominous a otros
        entornos, Sistemas Operativos o incluso consolas de videojuegos...
    \item Libertad para aprender mediante la lectura de la documentación del proyecto o del análisis del código.
    \item Facilidad para llegar a más usuarios gracias a la gratuidad del producto y a la disposición del mismo
        en Internet.
\end{itemize}

El acto de volcar todo el conocimiento de esta aplicación a la comunidad --- código, documentación, arte --- es desde
mi punto de vista un acto lógico y natural de intentar devolver a la comunidad del software libre una pequeña 
parte de todo lo que se ha obtenido de ella. Con un pequeño repaso a las herramientas utilizadas podemos comprobar
que la gran mayoría de ellas tienen licencias libres, y gracias a ellas hemos podido desarrollar esta aplicación con éxito,
pudiendo elegir entre un gran número de diferentes opciones y tomando la que mejor nos conviene en cada momento.


\subsubsection{Aplicación de lo aprendido en la carrera}

Existe una creencia extendida sobre que los estudios universitarios técnicos suelen pecar de facilitar conocimientos que
resultan muy teóricos, con muy poca aplicación práctica a la hora de desarrollar un proyecto real, así que es muy
gratificante poder comprobar que gracias todas las asignaturas de la carrera se consigue un fuerte bagaje y una metodología
de trabajo que nos permite llevar a cabo proyectos de gran envergadura como este. \\

La principal asignatura que nos ha permitido solventar los principales problemas de este proyecto ha sido la de Inteligencia
Artificial, impartida por el profesor Don Ignacio Pérez Blanquer, en especial el apartado de análisis de juegos. Los apartados
sobre representación del conocimiento y la base teórica sobre sistemas expertos también han resultado de gran ayuda.

Por otro lado, gracias a las asignaturas relacionadas con la Ingeniería del Software se han podido definir las líneas
maestras que gobiernan el desarrollo de la aplicación


\subsection{Aprendizaje del juego de dominó}

Lorem ipsum

\subsection{Primer proyecto que realizo de manera íntegra}

Lorem ipsum

\section{Trabajos futuros}

A la hora de establecer los requisitos técnicos de un proyecto fin de carrera se puede caer en la tentación de querer desarrollar y
especificar un proyecto que no sea realista con las limitaciones técnicas, de tiempo y de recursos que se disponen para la
realización del mismo. \\

Este acto puede resultar más flagrante en el caso de que el alumno se sienta muy atraído con el mismo --- como es el caso
que nos ocupa en este momento --- ya que el placer y el interés de desarrollar un producto profesional, similar a los que
se pueden encontrar en el mercado, puede llevarnos a una gran frustración por no conseguirlo. Debemos tener en cuenta de que
los estudios profesionales de videojuegos hoy en día cuentan con una gran cantidad de recursos, tanto de tiempo como
económicos, además de aglutinar profesionales de diferentes áreas (como puede ser el diseño gráfico, directores de arte,
expertos en usabilidad o programadores de bajo nivel, por nombrar algunos) y querer conseguir un producto de condiciones
similares es disparatado. \\

Es por estas razones que varias funcionalidades han debido de quedarse --- lamentablemente --- fuera del pliego de
especificaciones de requisitos. Aún así nos gustaría comentarlas a continuación, para que, por un lado, quede constancia
del trabajo de estudio preliminar del sector que se ha realizado y, por otro lado, por si alguien quiere recoger
el guante y continuar añadiendo nuevas funcionalidades al proyecto, mostrar un conjunto de pasos o características
muy interesantes de implementar:

\subsection{Juego en red}

lorem ipsum

\subsection{Servidor web para compartir sistemas expertos}

lorem ipsum

\subsection{Analizador de partidas con aprendizaje automático}

lorem ipsum

\subsection{Comunicación mediante servicios web}

lorem ipsum
