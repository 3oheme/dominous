% -*-cap2.tex-*-
% Este fichero es parte de la plantilla LaTeX para
% la realización de Proyectos Final de Carrera, protegido
% bajo los términos de la licencia GFDL.
% Para más información, la licencia completa viene incluida en el
% fichero fdl-1.3.tex

% Copyright (C) 2009 Pablo Recio Quijano 

\section{Resultados obtenidos}

Los resultados obtenidos gracias al desarrollo de esta aplicación han sido tremendamente satisfactorios, tanto en el
plano personal como en el profesional. \\

Como hitos conseguidos podemos destacar los siguientes:

\subsection{Primer simulador de dominó libre}

Sin querer resultar pretencioso --- y siempre según la investigación previa que hemos realizado --- Dominous es el primer
videojuego de dominó internacional bajo licencias libres, con todas las ventajas que ello conlleva:
\begin{itemize}
    \item Posibilidad de que pueda ser mejorado por la comunidad, añadiendo soluciones a errores de código o refactorizaciones,
        proporcionando nuevos jugadores, mejorando el sistema y motor de Inteligencia Artificial, añadiendo nuevas
        funcionalidades que se encuentran fuera de los requisitos iniciales del proyecto, portar Dominous a otros
        entornos, Sistemas Operativos o incluso consolas de videojuegos...
    \item Libertad para aprender mediante la lectura de la documentación del proyecto o del análisis del código.
    \item Facilidad para llegar a más usuarios gracias a la gratuidad del producto y a la disposición del mismo
        en Internet.
\end{itemize}

El acto de volcar todo el conocimiento de esta aplicación a la comunidad --- código, documentación, arte --- es desde
mi punto de vista un acto lógico y natural de intentar devolver a la comunidad del software libre una pequeña 
parte de todo lo que se ha obtenido de ella. Con un pequeño repaso a las herramientas utilizadas podemos comprobar
que la gran mayoría de ellas tienen licencias libres, y gracias a ellas hemos podido desarrollar esta aplicación con éxito,
pudiendo elegir entre un gran número de diferentes opciones y tomando la que mejor nos conviene en cada momento.



Aplicación de lo aprendido en la carrera. En especial, ampliación de los conceptos de IA estudiados ¿Llegaste a hacer la asignatura?.

Aprendizaje del juego de dominó.

Primer proyecto que realizo de manera íntegra.

\section{Trabajos futuros}

Juego en red.

Servidor web para compartir sistemas expertos.

Analizador de partidas con aprendizaje automático.

