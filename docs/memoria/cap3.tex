% -*-cap2.tex-*-
% Este fichero es parte de la plantilla LaTeX para
% la realización de Proyectos Final de Carrera, protejido
% bajo los términos de la licencia GFDL.
% Para más información, la licencia completa viene incluida en el
% fichero fdl-1.3.tex

% Copyright (C) 2009 Pablo Recio Quijano 

\section{Planificación}

Para el desarrollo de \textbf{Dominous} se decidió utilizar el modelo evolutivo iterativo incremental para el ciclo de
vida del proyecto. La decisión fue tomada ya que, a pesar de tener acotado el ámbito y los requisitos del programa,
la funcionalidad concreta de cada uno de los apartados se desconocía en un principio.

El modelo iterativo incremental es un modelo de tipo evolutivo que está basado en varios ciclos cascada realimentados aplicados
repetidamente, con una filosofía iterativa. Como se comenta en CITE, al final de cada iteración se le
realiza una entrega al cliente final; en este caso, el cliente ha sido el tutor del proyecto, que a cada iteración iba
dictando las líneas maestras generales a tomar a cada nuevo paso.

Las ventajas de utilizar un modelo iterativo incremental son básicamente los siguientes:
\begin{enumerate}
    \item Construir un sistema pequeño es siempre menos costoso en términos de riesgo.
    \item Al ir desarrollando parte de las funcionalidades, es más fácil determinar si los requerimientos
            planeados para los niveles subsiguientes son correctos.
    \item Si se comete algún error grave, sólo la última iteración necesita ser descartada.
    \item Reduciendo el tiempo de desarrollo de un sistema (en este caso en incremento del sistema) decrecen las
            probabilidades que esos requerimientos de usuarios puedan cambiar durante el desarrollo.
    \item Los errores de desarrollo realizados en un incremento, pueden ser arreglados antes del comienzo del próximo incremento.
\end{enumerate}

El proyecto \textbf{Dominous} consta de tres subsistemas que son los que ocupan el grueso del desarrollo:
\begin{enumerate}
    \item El primero es el motor de la partida: Controla los jugadores, las fichas en la mesa, la partida, situaciones irregulares
            y cualquier otro elemento referente únicamente al ámbito del dominó.
    \item Por otro lado está el motor gráfico, que será la interfaz entre la partida y el usuario, permitiendo el movimiento
            fluído por las diferentes secciones del programa e interactuando de forma directa con el motor de la partida,
            mostrando las fichas actuales y habilitando la interacción del jugador con el mundo.
    \item Y por último tenemos el motor de Inteligencia Artificial. Este motor será el que alimente la inteligencia
            y las acciones y decisiones de los jugadores controlados por el ordenador.
\end{enumerate}

\subsection{Incrementos realizados}

A continuación se citan los diferentes incrementos que se han ido realizando en el desarrollo del proyecto.

\subsubsection{Preliminares}

El primer paso a la hora de enfrentarse a un proyecto es decidir las herramientas que vamos a utilizar. En principio se
pensó utilizar el lenguage C++ por dos sencillas razones:
\begin{enumerate}
    \item Por un lado es un lenguaje que hemos aprendido en la carrera, se ha utilizado en varias asignaturas de
            diferentes ramas, con lo cual la comodidad y familiaridad que podemos tener a la hora de programar
            es un punto importante a tener en cuenta.
    \item Tampoco podemos olvidar que, al ser un lenguaje compilado, la velocidad de ejecución que se consigue
            es interesante, y mucho más tratándose de temas como la inteligencia artificial (donde puede ser
            necesario un uso intensivo de los recursos del sistema) o el desarrollo de videojuegos (en el que
            la potencia del ordenador repercute en una mejor experiencia del usuario)
\end{enumerate}
    
