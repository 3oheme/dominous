% ANÁLISIS

\section{Toma de requisitos}

En el desarrollo de esta aplicación la toma de requisitos se hizo mediante reuniones con el director del proyecto,
que realizaba el papel de cliente potencial de la misma. Después de varias reuniones se obtuvo el listado de requisitos
que se muestra en las siguientes secciones:

\subsection{Requisitos de interfaces externas}

En este apartado se van a describir los requisitos de conexión entre el software y el hardware, así como la interfaz
del usuario.\\

De la interfaz entre el software y el hardware se encarga la librería SDL, mediante el wrapper pygame --- y por encima
la capa que añade Gloss--- que, al ser un sistema preestablecido, no será necesario analizarlo ni diseñarlo,
simplemente haremos uso de él.\\

Así que pasamos a definir el interfaz entre el videojuego y el usuario. Todas las ventanas de la aplicación podrán
ser mostradas a pantalla completa o en formato de ventana con una resolución de 800 por 600 píxeles. A
continuación se definen las distintas ventanas con las que el usuario se puede encontrar:

\begin{description}
    \item[Ventana de introducción] Esta primera ventana mostrará únicamente el logotipo de Dominous, situando al usuario
            en contexto para iniciarlo en la ejecución del programa
    \item[Ventana del menú principal] La ventana del menú principal muestra el menú de inicio de Dominous, en el que
            el usuario podrá elegir entre las opciones más generales del juego, entre las que se encuentran:
            \begin{itemize}
                \item Partida clásica
                \item Laboratorio
                \item Opciones
                \item Tutorial
                \item Salir
            \end{itemize}
            Este menú y los siguientes que se describan serán completamente manejados por el ratón y bastará
            un clic encima de una opción para acceder a ella.
    \item[Ventana de selección de personaje] Esta ventana mostrará una interfaz que permite al usuario elegir los
            diferentes participantes que se enfrentarán en la siguiente partida. En caso del modo laboratorio
            se elegirán los cuatro jugadores, y en caso de partida clásica serán tres jugadores controlados por
            el ordenador más el jugador humano
    \item[Ventana de partida] Esta será la ventana principal de todo el juego. Mostrará una partida de dominó
            de dos equipos, el tablero e información de la partida, e irá actualizando el tablero según se vaya
            desarrollando la misma partida. Mediante la pulsación de la tecla ESC o clic sobre el botón menú se
            desplegará el menú interno de la partida, que permitirá abandonarla a pesar de no haberse terminado
            la partida actual.
    \item[Ventana de laboratorio] La ventana de laboratorio proporciona una interfaz para que el usuario de la
            aplicación pueda generar un conjunto elevado de partidas entre dos equipos definidos, con la idea
            de poder decidir qué pareja presenta las mejores características de Inteligencia Artificial.
    \item[Ventana del modo tutorial] Por último la ventana del modo tutorial mostrará al usuario información sobre
            el juego del dóminó mediante un conjunto de presentaciones, para que el mismo usuario pueda aprender
            más sobre el mundo del dominó sin necesidad de salir de la aplicación.
\end{description}

\subsection{Requisitos funcionales}
\subsection{Requisitos de rendimiento}
\subsection{Requisitos del sistema software}

La aplicación debe cumplir con los siguientes requisitos de sistema:
\begin{itemize}
    \item La aplicación debe ejecutarse de forma multiplafatorma, incluyendo como mínimo los sistemas operativos:
        \begin{itemize}
            \item En Microsoft Windows --- Realizándose las pruebas en la versión Windows 7 con las últimas actualizaciones
            \item En sistemas GNU/Linux --- Utilizando la distribución Ubuntu en su versión 10.04 con su instalación
                    por defecto y con todas las actualizaciones del sistema
        \end{itemize}
    \item El código de la aplicación no debe ser dependiente del sistema operativo en el que se desarrolle la aplicación,
            y debe ser un código mantenible y fácilmente ampliable para futuras mejoras y versiones
\end{itemize}
