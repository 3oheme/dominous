% ANÁLISIS

\section{Toma de requisitos}

En el desarrollo de esta aplicación la toma de requisitos se hizo mediante reuniones con el tutor del proyecto,
que realizaba el papel de cliente potencial de la misma. Después de varias reuniones se obtuvo el listado de requisitos
que se muestra en las siguientes secciones:


\subsection{Requisitos de interfaces externas}

En este apartado se van a describir los requisitos de conexión entre el software y el hardware, así como la interfaz
del usuario.\\

De la interfaz entre el software y el hardware se encarga la librería SDL, mediante el wrapper pygame --- y por encima
la capa que añade Gloss--- que, al ser un sistema preestablecido, no será necesario analizarlo ni diseñarlo,
simplemente haremos uso de él.\\

Así que pasamos a definir el interfaz entre el videojuego y el usuario. Todas las ventanas de la aplicación podrán
ser mostradas a pantalla completa o en formato de ventana con una resolución de 800 por 600 píxeles. A
continuación se definen las distintas ventanas con las que el usuario se puede encontrar:

\begin{description}
    \item[Ventana de introducción] Esta primera ventana mostrará únicamente el logotipo de Dominous, situando al usuario
            en contexto para iniciarlo en la ejecución del programa
    \item[Ventana del menú principal] La ventana del menú principal muestra el menú de inicio de Dominous, en el que
            el usuario podrá elegir entre las opciones más generales del juego, entre las que se encuentran:
            \begin{itemize}
                \item Partida clásica
                \item Laboratorio
                \item Opciones
                \item Tutorial
                \item Salir
            \end{itemize}
            Este menú y los siguientes que se describan serán completamente manejados por el ratón y bastará
            un clic encima de una opción para acceder a ella.
    \item[Ventana de selección de personaje] Esta ventana mostrará una interfaz que permite al usuario elegir los
            diferentes participantes que se enfrentarán en la siguiente partida. En caso del modo laboratorio
            se elegirán los cuatro jugadores, y en caso de partida clásica serán tres jugadores controlados por
            el ordenador más el jugador humano.
    \item[Ventana de partida] Esta será la ventana principal de todo el juego. Mostrará una partida de dominó
            de dos equipos, el tablero e información de la partida, e irá actualizando el tablero según se vaya
            desarrollando la misma partida. Mediante la pulsación de la tecla ESC o clic sobre el botón menú se
            desplegará el menú interno de la partida, que permitirá abandonarla a pesar de no haberse terminado
            la partida actual.
    \item[Ventana de laboratorio] La ventana de laboratorio proporciona una interfaz para que el usuario de la
            aplicación pueda generar un conjunto elevado de partidas entre dos equipos definidos, con la idea
            de poder decidir qué pareja presenta las mejores características de Inteligencia Artificial.
    \item[Ventana del modo tutorial] Por último la ventana del modo tutorial mostrará al usuario información sobre
            el juego del dóminó mediante un conjunto de presentaciones, para que el mismo usuario pueda aprender
            más sobre el mundo del dominó sin necesidad de salir de la aplicación.
\end{description}


\subsection{Requisitos funcionales}

Los requisitos funcionales que el sistema debe ofrecer al usuario son los siguientes:
\begin{itemize}
    \item Poder jugar una partida de dominó con tres jugadores más controlados por el ordenador.
    \item Enfrentar a dos parejas de jugadores controlados por ordenador, haciendo que jueguen un gran número
            de partidas seguidas en modo automático (esto es, sin visualizar la partida que se desarrolla y
            mostrando únicamente las victorias), con la finalidad de poder decidir qué pareja posee una
            inteligencia artificial más avanzada.
    \item Acceder al modo tutorial, para realizar un aprendizaje de las normas, técnicas y usos del dominó.
    \item Cambiar el tipo de juego para que cuatro jugadores manejados por la máquina puedan desarrollar una
            partida en modo visual.
    \item Seleccionar otro tema gráfico para que, tanto fichas como tablero como otros elementos gráficos, cambien
            a gusto del usuario, eligiendo entre cierto abanico de temas.
    \item Cambiar de modo ventana a modo pantalla completa.
\end{itemize}


\subsection{Requisitos de rendimiento}

El rendimiento de la aplicación debe ser tal que permita un desempeño agradable de la partida. Este requisito
hace referencia principalmente a los siguientes asuntos:
\begin{itemize}
    \item El sistema de inteligencia artificial debe ser lo suficientemente ágil y estar ajustado y
            perfeccionado para que los tiempos empleados en cálculos de toma de decisiones no ralenticen
            la partida. Se cuenta como asunto el que, en el desarrollo de una partida de dominó, los tiempos de
            espera también se interpretan, por lo que debemos realizar los cálculos dentro de un cierto
            margen de tiempo.
    \item Por otro lado, el motor gráfico debe estar optimizado para que el usuario no aprecie movimientos
            bruscos a la hora de manejar la aplicación. No olvidemos que estamos desarrollando un videojuego,
            así que el programa debe mostrar cierta agilidad a la hora de realizar movimientos y transiciones
            entre los diferentes estados de la partida, incluyendo menús, fichas, o asuntos relativos a la
            interfaz, como puede ser el arrastrar una ficha a su lugar correspondiente.
\end{itemize}

Es importante recordar en todo momento que estamos desarrollando una aplicación en tiempo real, por lo que
debe primar la velocidad sobre otros factores como el consumo de memoria principal.


\subsection{Restricciones de diseño}
Como bien comentábamos en el punto anterior, a la hora de realizar el diseño de la aplicación tienen que
primar los tiempos de respuesta sobre el consumo de recursos de la memoria principal o secundaria. Esta
es la principal restricción que tendrá el diseño de nuestra aplicación.\\

Los videojuegos están pensados para ejecutarse como aplicación principal, no para compartir recurssos
con otros programas; por esta razón se permite que consuman muchos recursos.


\subsection{Requisitos del sistema software}

La aplicación debe cumplir con los siguientes requisitos de sistema:
\begin{itemize}
    \item La aplicación debe ejecutarse de forma multiplafatorma, incluyendo como mínimo los sistemas operativos:
        \begin{itemize}
            \item En Microsoft Windows --- Realizándose las pruebas en la versión Windows 7 con las últimas actualizaciones
            \item En sistemas GNU/Linux --- Utilizando la distribución Ubuntu en su versión 10.04 con su instalación
                    por defecto y con todas las actualizaciones del sistema.
        \end{itemize}
    \item El código de la aplicación no debe ser dependiente del sistema operativo en el que se desarrolle la aplicación,
            y debe ser un código mantenible y fácilmente ampliable para futuras mejoras y versiones.
\end{itemize}

\section{Modelo de casos de uso}

Para describir los comportamientos que tendrá el sistema, utilizaremos el lenguaje guaje de modelado de sistemas UML;
éste representa los requisitos funcionales de todo el sistema, centrándose en qué hace pero no en cómo lo hace.\\

A continuación describimos uno por uno cada caso de uso.

\subsection{Diagrama de casos de uso}

Como primer paso, debemos mostrar el diagrama de casos de uso que representa la funcionalidad completa de la aplicación.
El esquema utilizado es el siguiente:
\begin{enumerate}
    \item Identificar los usuarios del sistema y sus posibles roles.
    \item Para cada rol definido, identificar todas las formas que tiene de interactuar con el sistema. En el caso
            de Dominous, existe un único rol de acceso a la aplicación, por lo que la especificación de usuario
            será única.
    \item Crear todos los casos de uso para poder describir los objetivos que se desean cumplir.
    \item Estructurar y definir esos casos de uso.
\end{enumerate}

\begin{figure}[h]
  \label{diagrama-casos-uso}
  \begin{center}
    \includegraphics[scale=0.7]{diagrama_casos_de_uso.png}
  \end{center}
  \caption{Diagrama de casos de uso del sistema}
\end{figure}


\subsection{Descripción de los casos de uso}

A continuación pasamos a la descripción de los casos de uso. Para ello se va a utilizar una notación formal
usando plantillas, con la intención y finalidad de que este texto sea legible y comprensible por
un usuario que no sea experto.

\subsubsection{Caso de uso: Menú principal}

\begin{description}
    \item[Caso de uso] Menú principal
    \item[Descripción] Se muestra el menú principal de la aplicación, desde donde es posible acceder a los
        diferentes modos de juego y a las opciones.
    \item[Actores] Usuario
    \item[Precondiciones] Ninguna
    \item[Postcondiciones] Ninguna
    \item[Escenario principal]
        \begin{enumerate}
            \item El sistema muestra el menú principal del juego en la pantalla
            \item El usuario selecciona el modo \textbf{partida simple}
            \item El sistema inicia el modo de elección de jugadores
        \end{enumerate}
    \item[Extensiones --- flujo alternativo] $\quad$
        \begin{description}
            \item[*a ] El usuario cierra la ventana de la aplicación y sale de la aplicación
            \item[2a ] El usuario pulsa sobre el botón de laboratorio, dirigiéndose a ese apartado de la aplicación
            \item[2b ] El usuario pulsa el botón de tutorial, dirigiéndose a ese apartado de la aplicación
            \item[2c ] El usuario pulsa sobre las opciones, dirigiéndose a ese apartado de la aplicación
            \item[2d ] El usuario pulsa sobre el botón de salir, cerrándose la aplicación
        \end{description}
   
\end{description}

\subsubsection{Caso de uso: Salir}

\begin{description}
    \item[Caso de uso] Salir
    \item[Descripción] Primeramente se muestra la pantalla de información de la aplicación --- con más datos
            sobre desarrolladores, licencias y cualquier otra información que pueda resultar de interés para
            el usuario.
    \item[Actores] Usuario.
    \item[Precondiciones] Ninguna.
    \item[Postcondiciones] Se sale de la aplicación.
    \item[Escenario principal]
        \begin{enumerate}
            \item El sistema muestra la pantalla de información
            \item El usuario pulsa sobre cualquier lugar de la aplicación
            \item La aplicación se cierra
        \end{enumerate}
    \item[Extensiones --- flujo alternativo] $\quad$
        \begin{description}
            \item[*a ] El usuario cierra la ventana de la aplicación y sale de la aplicación
        \end{description}
\end{description}
 

\subsubsection{Caso de uso: Partida simple}
\subsubsection{Caso de uso: Laboratorio}
\subsubsection{Caso de uso: Opciones}
\subsubsection{Caso de uso: Selección de personaje}
