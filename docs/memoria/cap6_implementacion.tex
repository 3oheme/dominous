% -*-cap2.tex-*-
% Este fichero es parte de la plantilla LaTeX para
% la realización de Proyectos Final de Carrera, protegido
% bajo los términos de la licencia GFDL.
% Para más información, la licencia completa viene incluida en el
% fichero fdl-1.3.tex

% Copyright (C) 2009 Pablo Recio Quijano 

\section{Implementación}

Una vez terminadas las dos fases previas --- análisis y diseño --- es hora de enfrentarnos a la codificación e implementación
en sí de la aplicación. Este apartado es largo y entretenido, y va descubriendo a cada momento nuevas problemáticas que
no se han tenido en cuenta antes y es necesario resolver, pero también es apasionante, porque vamos viendo de forma
empírica cómo nuestra aplicación va tomando forma y se va haciendo tangible a cada momento que avanza. \\

El desarrollo de Dominous ha presentado muchas características interesantes, problemas, dificultades y otras circunstancias
que creemos interesante destacar, y que se pasamos a comentar a continuación.

\subsection{Entorno gráfico}

El entorno gráfico se puede dividir en dos apartados de diferente complejidad: por un lado tenemos la gestión de menús,
opciones, pantallas y secciones de la aplicación, y por otro lado tenemos la gestión de una partida de dominó, con todas
sus interacciones, animaciones y movimientos de fichas, eventos y demás.

\subsubsection{Sistema de secciones}

Todo el sistema de secciones lo gobierna el objeto principal de la aplicación, llamado \textbf{dominous}. Este objeto posee
diferentes objetos como atributos, y estos atributos son las diferentes secciones de la aplicación. Cuando queremos
que el flujo de la aplicación pase de una sección a otra no tenemos más que llamarla; al llamarla estaremos ejecutando
el siguiente fragmento:

\begin{lstlisting} [language=Python, numbers=left]
def goto_tutorial(self):
    self.flow.stop()
    self.flow = self.tutorial
    self.flow.start()
\end{lstlisting}

Gracias a este tipo de llamadas podemos subdividir cómodamente la aplicación en módulos muy acotados, a los cuales se aplica
primero una función de inicialización --- por si es necesario que actualicen estructuras, datos o cualquier otra labor
de preparación que puedan requerir --- y una vez se termina también se ejecuta una función de finalización. \\

Y mientras el flujo esté asociado a una sección, todas las funciones de eventos, dibujado y actualización de la pantalla
se disparan dentro de la sección correspondiente. \\

Gracias a esta metodología hemos podido acotar de forma muy cómoda las diferentes secciones (cada una con sus peculiaridades)
y gestionar de forma independiente cada apartado, evitando posibles problemas colaterales, asegurando una buena
integración y minimizando el hecho de que errores de un apartado puedan afectar a otros. \\


\subsubsection{Gestión de partida}

Lorem ipsum


\subsection{Interfaz de sonido}

La interfaz de sonido se ha diseñado de tal forma que se gestione de forma independiente para que su uso sea sencillo,
realizando llamadas al objeto \textbf{sound} con los métodos que estimemos oportunos en cada momento, y el propio
sistema se encarga de realizar diferentes cálculos y comprobaciones:

\begin{itemize} 
    \item Por ejemplo, a la hora de ejecutar un determinado fragmento musical, controla si ya se está escuchando música
        en ese momento; si es ese el caso realiza un pequeño fundido de la pista actual, e inicia la ejecución de la
        nueva, evitando así los desagradables clicks que se podrían escuchar cuando se para y se inicia la canción.
    \item O también, para ejecutar un sonido de ficha de dominó sobre el tablero, el sistema realiza la precarga de todos
        los sonidos --- similares pero no iguales --- y dispara uno aleatoriamente, evitando que cada apartado tenga
        que gestionarlo de forma autónoma
\end{itemize}


\subsection{Configuración de la aplicación}

La configuración de la aplicación (y almacenaje de la misma) se realiza mediante ficheros \textbf{ini}

\subsection{Inteligencia Artificial}

Dolor sit amet
