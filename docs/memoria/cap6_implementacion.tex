% -*-cap2.tex-*-
% Este fichero es parte de la plantilla LaTeX para
% la realización de Proyectos Final de Carrera, protegido
% bajo los términos de la licencia GFDL.
% Para más información, la licencia completa viene incluida en el
% fichero fdl-1.3.tex

% Copyright (C) 2009 Pablo Recio Quijano 

\section{Implementación}

Una vez terminadas las dos fases previas --- análisis y diseño --- es hora de enfrentarnos a la codificación e implementación
en sí de la aplicación. Este apartado es largo y entretenido, y va descubriendo a cada momento nuevas problemáticas que
no se han tenido en cuenta antes y es necesario resolver, pero también es apasionante, porque vamos viendo de forma
empírica cómo nuestra aplicación va tomando forma y se va haciendo tangible a cada momento que avanza. \\

El desarrollo de Dominous ha presentado en diferentes momentos varios problemas y dificultades que creemos interesante destacar,
y que se pasamos a comentar a continuación.

\subsection{Entorno gráfico}

El entorno gráfico se puede dividir en dos apartados de diferente complejidad: por un lado tenemos la gestión de menús,
opciones, pantallas y secciones de la aplicación, y por otro lado tenemos la gestión de una partida de dominó, con todas
sus interacciones, animaciones y movimientos de fichas, eventos y demás.

\subsubsection{Sistema de secciones}

Todo el sistema de secciones lo gobierna el objeto principal de la aplicación, llamado \textbf{dominous}. Este objeto posee
diferentes objetos como atributos, y estos atributos son las diferentes secciones de la aplicación. Cuando queremos
que el flujo de la aplicación pase de una sección a otra no tenemos más que llamarla; al llamarla estaremos ejecutando
el siguiente fragmento

\begin{lstlisting}[caption={[]Cambio de flujo de la aplicación}, numbers=left]

def goto_tutorial(self):
    self.flow.stop()
    self.flow = self.tutorial
    self.flow.start()

\end{lstlisting}

\subsubsection{Gestión de partida}

Lorem ipsum

\subsection{Inteligencia Artificial}

Dolor sit amet
