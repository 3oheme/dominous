% -*-cap2.tex-*-
% Este fichero es parte de la plantilla LaTeX para
% la realización de Proyectos Final de Carrera, protegido
% bajo los términos de la licencia GFDL.
% Para más información, la licencia completa viene incluida en el
% fichero fdl-1.3.tex

% Copyright (C) 2009 Pablo Recio Quijano 

% MANUAL PARA AÑADIR NUEVOS TEMAS GRAFICOS

\section{Nuevas posibilidades gráficas}

Dominous se ha diseñado teniendo en mente que el sistema debe ser extensible en varias dimensiones. Antes hemos
explicado que es posible crear nuevos jugadores, dotarlos de una inteligencia artificial adaptada a nuestras necesidades
e incluso asignarles un avatar que los identifique. Pero además de oponentes, podemos dar un paso más en la
personalización de nuestra aplicación y crear nuevos temas gráficos, que nos permita sentirnos cómodos dentro de la
aplicación y disfrutar del juego de forma más cómoda. \\

\section{Estructura de directorios}

Para crear un nuevo tema gráfico no es necesario modificar ningún fichero de configuración, simplemente debemos crear una
estructura de directorios y un conjunto de imágenes gráficas con un formato que --- suponiendo que estamos desarrollando
un nuevo tema gráfico de nombre \textbf{mytheme} --- se describe a continuación :

\begin{lstlisting} [language=Python, numbers=left]
\themes
  \mytheme
    \images
      \background
      \gui
      \tiles
    \sounds
      \ambient
      \music
      \tiles
\end{lstlisting}


